\documentclass{article}

\begin{document}

% todo: this section will cover the idea of a design space of TMCs with a given set of ligands, how the space scales and how we can compare different TMCs
% - exposition of a combinatorial view of TMC space
% - comments on symmetry classes and how they effect the dimensions of the search space
% - spectrochemical series, other papers about ligand field theory

\section{Size of transition metal complex (TMC) space}
Among the most extensively analyzed chemical subspaces are the organic ones. Due to the graph theoretically tractable nature of carbon scaffolds enumeration is much easier than in other subspaces. The usual number of molecules in the chemical space of organic molecules with less than 500 Da is estimated to be $10^{60}$.\ref{blair1932,polishchuk2013,bohacek1998,wester2008,triggle2009} Including larger molecules and materials from the whole periodic table ends up in a vastly larger number of molecules still.

For enumeration, it is important to find the right representation of molecules.\ref{bartok2013,ghiringhelli2015} Computationally generated data sets consist of the molecule's identity and a number of descriptors. Chemical space can be defined as a Cartesian space in the dimension of the number of the features. Therefore, each set of descriptors spans chemical space in a different way including some molecules that possibly overlap if the descriptor set is not diverse enough. Our descriptors called RACs are introduced in section X.
 
The largest databases of existing molecules are not only just a fraction of the actual space but is also heavily biased towards easily accessible molecules through synthesis.\ref{tan2005,hajduk2011,galloway2010} Computational high-throughput screening\ref{hachmann2011, jain2011, hautier2010, jensen2015, norskov2009, greeley2006, curtarolo2013, rajan2008} is a potential remedy but severly constraint by computational cost.\ref{kirkpatrick2004} Enumeration projects of computationally generated molecules have attempted to either exhaust or systematically cover\ref{virshup2013} large subspaces. Most prominently, the GDB-17 dataset\ref{ruddigkeit2012} tries to enumerate all possible organic scaffold based structures of up to 17 atoms of C, N, O, S, and halogens. This results in approximately 166 billion organic small molecules, exhibiting more diversity than other data sets and lead to new discoveries.\ref{luethi2010,nguyen2008} Instead of going through all possibilities of structures, Virshup \textit{et al.}\ref{virshup2013} introduced an algorithm to stochastically\ref{agrafiotis1997} sample the chemical space for a so-called representative sublibrary, which is defined as representative subspace with a smaller amount of molecules than the parent space without losing it's diversity.

TMCs form promising functional inorganic materials due to their wide range of tunable electronic properties. They are crucial for contemporary challenges, such as spincrossover complexes,\ref{letard2004,halcrow2011,ashley2017} dye-sensitizers in solar cells,\ref{bignozzi2013} or open-shell catalysts.\ref{harvey2003} Nonetheless, few benchmark data sets, experimental data bases or softwares are available. In this group's research program, we try to systematically explore the TMC space.  However, exhaustive enumeration and calculation of all possible ligand fields is as intractable as in organic chemistry. For this, the open-source software molSimplify\ref{ioannidis2016} was developed for the rapid structure generation in coordination chemistry. 

\section{The role of symmetry}



\section{Spectrochemical Series}




\end{document}